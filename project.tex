\documentclass[a4paper,12pt]{article}
\frenchspacing
\usepackage[utf8]{inputenc}
\usepackage{a4wide}
%\usepackage[finnish]{babel}
\usepackage[english]{babel}
\usepackage{mathtools}
\usepackage{siunitx}
%\usepackage[utf8]{inputenc}
\usepackage[pdftex]{graphicx}
\usepackage{icomma}
\usepackage{hyperref}
\usepackage{amssymb}
\usepackage{float}
%\usepackage{mhchem}
\usepackage{parskip}
\usepackage{graphicx}
\usepackage{caption}
\usepackage{subcaption}
\usepackage{fancyhdr}
\usepackage{eurosym}
\usepackage{enumerate}
%\usepackage{subfig}
%\usepackage{floatrow}
%\floatsetup[figure]{style=plain,subcapbesideposition=top}



%%% NEW COMMANDS

\newcommand{\dd}{\,\mathrm{d}}
\newcommand{\exerline}{
\vspace*{.1cm}
\noindent \rule{\textwidth}{1pt}
\vspace*{.1cm}
}

%\usepackage{lastpage}

\pagestyle{fancy}
\fancyhead{}
\fancyhead[LO,RE]{Comp. Phys. -- Project}
%\fancyhead[RO,LE]{Harjoitus 1, 14.9.2012}
\fancyhead[RO,LE]{Kohvakka, 2019}
%\fancyfoot{}
%\fancyfoot[LO,RE]{Kangaslampi / Laaksonen}
%\fancyfoot[RO,LE]{\thepage/\pageref{LastPage}}


\hyphenation{every-where}
\renewcommand{\baselinestretch}{1}

\begin{document}


%\begin{minipage}[t][1.5cm][b]{.2\textwidth}
%\AaltoLogoRandomLarge{0.7}
%\end{minipage}
\begin{minipage}[t][1.5cm][b]{\textwidth}
\begin{center}
\Large{\textbf{Computational Physics}} \\
\vspace*{.1cm}
\Large{\textbf{Ex. 12 -- DFT solver}}\\
\vspace*{.1cm}
\large{Kassius Kohvakka, 586977}
\end{center}
\end{minipage} 
\vspace{-0.4cm}

\exerline

The exercise was done by modifying the provided Matlab code. The solver is implemented in \texttt{DFT\_solver.m}. The script \texttt{run\_DFT\_solver.m} was used to run the solver and plot the results.

\textbf{(a)} The resulting electron density for $M=1$ nucleus and $N=3$ electrons is shown in Fig.~\ref{fig: density}. The normalization of the solver seems to be correct as $\int \rho (x) \dd x = 3$, as it should be since the normalized basis functions $\phi_i$ should satisfy

\begin{equation}
\int_\infty^\infty \rho (x) \dd x = \int_\infty^\infty \sum_{i=1}^{3} |\phi_i(x)|^2 \dd x = \sum_{i=1}^{3} \int_\infty^\infty |\phi_i(x)|^2 \dd x = \sum_{i=1}^{3} 1 = 3.
\end{equation}

The resulting electron density seems to be well contained within the potential well of the nucleus, which seems plausible.

\textbf{(b)} The system with two nuclei ($M=2$) and six electrons ($N=6$) was set up and the DFT solver was run at varying inter-nucleus separations $\Delta x$. The total energy of the system as a  function of the separation is shown in Fig.~\ref{fig: energy}. From the plot, we can clearly see that the total energy is minimized by $\Delta x \approx 0.5$, indicating the presence of simple chemical bonding in our DFT system. The electron density for the energy-minimizing separation is plotted in Fig.~\ref{fig: min_energy}, along with the locations of, and the external potential caused by the two nuclei.

\textbf{(c)} About 8 hours in all.

%\begin{figure}[h]
%\centering
%\includegraphics[width=0.7\textwidth]{figs/minimum_energy.pdf}
%\caption{The electron density at the separation $\Delta x$ minimizing the total energy of the system with $M=2$, $N=6$. The vertical lines indicate the locations of the two nuclei. The external potential due to the nuclei is shown in arbitrary units with the dashed line.}
%\label{fig: min_energy}
%\end{figure}

%\begin{figure}[h]
%\centering
%\begin{subfigure}[b]{.5\textwidth}
%  \centering
%  \includegraphics[width=\textwidth]{figs/wave_euler_forward.pdf}
%  \caption{Implicit Euler}
%  \label{fig:sub1}
%\end{subfigure}%
%\begin{subfigure}[b]{.5\textwidth}
%  \centering  
%  \includegraphics[width=\textwidth]{figs/wave_cn_forward.pdf}
%  \caption{Crank-Nicolson}
%  \label{fig:sub2}
%\end{subfigure}
%\caption{The results for the two integration schemes for forward time integration of the wave equation from $t_0=0$ to $t_{f}$ = 0.2}
%\label{fig:waveForward}
%\end{figure}



\end{document}
